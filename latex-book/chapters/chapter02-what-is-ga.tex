\chapter{What is a Genetic Algorithm?}

\section{Introduction}
Genetic Algorithms (GAs) are search \begin{table}[H]
\centering
\begin{tabular}{cccc}
\toprule
Individual & Binary & Decimal & Fitness \\
\midrule
1 & 01101 & 13 & 169 \\
2 & 11000 & 24 & 576 \\
3 & 01000 & 8 & 64 \\
4 & 10011 & 19 & 361 \\
\bottomrule
\end{tabular}
\caption{Initial Population Example}
\end{table}t mimic the process of natural selection~\cite{holland1975adaptation, goldberg1989genetic, mitchell1996introduction}. They belong to the larger class of evolutionary algorithms and are particularly useful for optimization and search problems.

\section{Biological Inspiration}
GAs are inspired by Charles Darwin's theory of natural evolution. In nature:
\begin{itemize}
    \item Individuals with better fitness have higher chances of survival
    \item Successful traits are passed to offspring through reproduction
    \item Genetic diversity is maintained through mutation
    \item Population evolves over generations toward better adaptation
\end{itemize}

\section{Basic Terminology}
\subsection{Genetic Algorithm Terms}
\begin{itemize}
    \item \textbf{Individual/Chromosome}: A candidate solution
    \item \textbf{Gene}: A single element of an individual
    \item \textbf{Allele}: The value of a gene
    \item \textbf{Population}: Collection of individuals
    \item \textbf{Generation}: One iteration of the algorithm
    \item \textbf{Fitness}: Quality measure of an individual
    \item \textbf{Genotype}: Encoded representation of a solution
    \item \textbf{Phenotype}: Decoded representation of a solution
\end{itemize}

\section{Basic Structure of a Genetic Algorithm}

\begin{algorithm}
\caption{Basic Genetic Algorithm}
\begin{algorithmic}
\STATE Initialize population randomly
\WHILE{termination condition not met}
    \STATE Evaluate fitness of each individual
    \STATE Select parents for reproduction
    \STATE Apply crossover to create offspring
    \STATE Apply mutation to offspring
    \STATE Select survivors for next generation
    \STATE Increment generation counter
\ENDWHILE
\STATE Return best individual found
\end{algorithmic}
\end{algorithm}

\section{Key Components of GA}

\subsection{Representation}
The representation defines how solutions are encoded:
\begin{itemize}
    \item \textbf{Binary representation}: Solutions encoded as binary strings
    \item \textbf{Real-valued representation}: Solutions as real numbers
    \item \textbf{Permutation representation}: Solutions as ordered sequences
    \item \textbf{Tree representation}: Solutions as tree structures
\end{itemize}

\subsection{Fitness Function}
The fitness function evaluates the quality of each individual:
\begin{equation}
fitness(x) = f(x)
\end{equation}

For maximization problems, higher fitness values are better. For minimization problems, fitness is often defined as:
\begin{equation}
fitness(x) = \frac{1}{1 + f(x)}
\end{equation}

\subsection{Selection}
Selection determines which individuals become parents:
\begin{itemize}
    \item \textbf{Roulette Wheel Selection}: Probability proportional to fitness
    \item \textbf{Tournament Selection}: Best individual from random subset
    \item \textbf{Rank Selection}: Selection based on fitness ranking
\end{itemize}

\subsection{Crossover (Recombination)}
Crossover combines genetic material from two parents:
\begin{itemize}
    \item \textbf{One-point crossover}: Single crossover point
    \item \textbf{Two-point crossover}: Two crossover points
    \item \textbf{Uniform crossover}: Random selection from parents
\end{itemize}

\subsection{Mutation}
Mutation introduces random changes to maintain diversity:
\begin{itemize}
    \item \textbf{Bit-flip mutation}: For binary representation
    \item \textbf{Gaussian mutation}: For real-valued representation
    \item \textbf{Swap mutation}: For permutation representation
\end{itemize}

\section{Example: Maximizing a Simple Function}

Consider maximizing $f(x) = x^2$ where $x \in [0, 31]$.

\subsection{Step 1: Representation}
Use 5-bit binary strings: $x = 10110_2 = 22_{10}$

\subsection{Step 2: Initial Population}
\begin{table}[H]
\centering
\begin{tabular}{cccc}
\toprule
Individual & Binary & Decimal & Fitness \\
\midrule
1 & 01101 & 13 & 169 \\
2 & 11000 & 24 & 576 \\
3 & 01000 & 8 & 64 \\
4 & 10011 & 19 & 361 \\
\bottomrule
\end{tabular}
\caption{Initial Population Example}
\end{table}

\subsection{Step 3: Selection}
Select individuals 2 and 4 (highest fitness) as parents.

\subsection{Step 4: Crossover}
Parents: 11000 and 10011
Crossover point: position 3
Offspring: 11011 (27) and 10000 (16)

\subsection{Step 5: Mutation}
Apply bit-flip mutation with low probability.

\section{Advantages of Genetic Algorithms}
\begin{itemize}
    \item \textbf{Global search}: Can escape local optima~\cite{goldberg1989genetic}
    \item \textbf{Parallelizable}: Population-based approach~\cite{eiben2015introduction}
    \item \textbf{Flexible}: Applicable to various problem types~\cite{sivanandam2008introduction}
    \item \textbf{No gradient required}: Works with discontinuous functions~\cite{mitchell1996introduction}
    \item \textbf{Robust}: Handles noisy fitness functions~\cite{haupt2004practical}
\end{itemize}

\section{Disadvantages of Genetic Algorithms}
\begin{itemize}
    \item \textbf{Computational cost}: May require many function evaluations
    \item \textbf{Parameter tuning}: Many parameters to set
    \item \textbf{No guarantee}: May not find global optimum
    \item \textbf{Premature convergence}: Population may lose diversity
\end{itemize}

\section{When to Use Genetic Algorithms}
GAs are particularly suitable when:
\begin{itemize}
    \item Search space is large and complex
    \item Little is known about the problem structure
    \item Traditional methods fail or are inappropriate
    \item Multiple objectives need to be optimized
    \item Robustness is more important than efficiency
\end{itemize}

\section{Variations of Genetic Algorithms}
\begin{itemize}
    \item \textbf{Steady-state GA}: Replace one individual per generation
    \item \textbf{Parallel GA}: Multiple populations evolve simultaneously
    \item \textbf{Hybrid GA}: Combine with local search methods
    \item \textbf{Multi-objective GA}: Handle multiple objectives
\end{itemize}

\section{Chapter Summary}
This chapter introduced genetic algorithms as optimization tools inspired by natural evolution. We covered the basic components, terminology, and a simple example. The key insight is that GAs use population-based search with selection, crossover, and mutation to evolve solutions toward optimality.

\section{Key Concepts}
\begin{itemize}
    \item Biological inspiration and evolution metaphor
    \item Basic GA structure and components
    \item Representation, fitness, selection, crossover, mutation
    \item Advantages and limitations of GAs
    \item When to apply genetic algorithms
\end{itemize}