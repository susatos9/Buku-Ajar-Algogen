\chapter{Crossover (Rekombinasi) dalam Algoritma Genetika}

\section{Pendahuluan Crossover}

Secara biologis, crossover terinspirasi oleh reproduksi seksual di mana kromosom saling bertukar segmen selama meiosis dan gen-gen tersebar secara independen ke dalam gamet. Mekanisme alami ini menghasilkan variasi: keturunan berbeda dari induknya, yang meningkatkan keanekaragaman dan menyediakan bahan mentah bagi seleksi. Analogi ini menekankan dua gagasan berguna untuk algoritma: mencampurkan materi induk untuk mempertahankan dan menggabungkan substruktur yang menguntungkan, serta memperkenalkan variasi untuk menghindari konvergensi prematur.

Fenomena utama dari biologi dapat dipetakan langsung ke desain algoritma. Crossing over menukar segmen-segmen genetik bersebelahan (mempertahankan struktur lokal), independent assortment merandomisasi kombinasi kromosom (mendorong campuran baru), dan keanekaragaman genetik yang dihasilkan membantu populasi keluar dari optimal lokal. Gagasan "blok penyusun" (building blocks) menunjukkan bahwa kombinasi gen yang kompak dan berkualitas tinggi harus dilindungi dan dikombinasikan kembali, bukan dihancurkan oleh operator yang terlalu merusak.

Secara konseptual, proses crossover biasanya melalui tiga langkah: memilih pasangan induk untuk kawin (biasanya dipengaruhi oleh fitness), memilih satu atau lebih titik crossover atau skema rekombinasi, dan menukar materi genetik untuk membentuk keturunan. Detail implementasi (di mana titik pemisah ditempatkan atau apakah gen digabung secara aritmetika) menentukan seberapa banyak struktur yang dipertahankan dibandingkan seberapa banyak kebaruan yang diperkenalkan; pilihan-pilihan ini secara kritis membentuk dinamika pencarian.

Crossover memediasi trade-off antara eksplorasi dan eksploitasi. Eksploitasi muncul ketika operator berhasil menggabungkan dan mempertahankan substruktur yang baik dari induk, mempercepat kemajuan menuju solusi berkualitas tinggi. Eksplorasi terjadi ketika rekombinasi menghasilkan konfigurasi baru yang tidak ada pada kedua induk, sehingga meningkatkan peluang menemukan wilayah ruang pencarian yang lebih baik. Desain algoritma yang efektif mengatur operator sehingga kedua perilaku ini terjadi secara seimbang.

Risiko disrupsi skema—memecah kombinasi gen yang berguna—tergantung pada bagaimana crossover diimplementasikan dan di mana potongan terjadi. Skema yang kurang merusak mempertahankan hubungan bersebelahan dan skema pendek, sementara skema yang lebih merusak (atau banyak titik pemotongan) dapat menghancurkan pola gen panjang yang terkoadaptasi meskipun meningkatkan keanekaragaman. Kesadaran akan efek-efek ini memandu pemilihan gaya crossover dan parameternya untuk domain masalah tertentu.

\section{Operator Crossover untuk Kromosom Biner}

\subsection{Definisi dan Fungsi Operator Crossover}

Crossover adalah operator genetik yang digunakan untuk memvariasikan susunan kromosom dari satu generasi ke generasi berikutnya~\cite{geeksforgeeks_crossover, tutorialspoint_crossover, course_week4_crossover}. Metode crossover yang digunakan tergantung pada metode pengkodean yang dipakai.

Dalam praktiknya, crossover terjadi dalam tiga tahapan: operator reproduksi memilih sepasang individu untuk kawin, sebuah titik crossover dipilih sepanjang panjang string, dan nilai-nilai setelah titik tersebut ditukar antara dua induk untuk membentuk keturunan baru.

Dalam representasi kromosom biner, setiap individu dalam populasi direpresentasikan sebagai urutan bit (0 dan 1) yang mengekspresikan solusi potensial untuk suatu masalah.

\subsection{Crossover Satu Titik}

Crossover satu titik adalah bentuk paling sederhana dari crossover. Satu posisi crossover dipilih secara acak, kemudian bagian dari kedua induk setelah posisi tersebut ditukar untuk membentuk dua keturunan baru. String hasil dari proses ini memiliki karakteristik bias posisi (positional bias).

\begin{figure}[h]
\centering
\includegraphics[width=0.7\textwidth]{figures/buku_ajar_page_24.png}
\caption{Crossover Satu Titik untuk kromosom biner}
\label{fig:binary_single_point}
\end{figure}

\subsection{Crossover Satu Titik}

Crossover satu titik membagi kromosom menjadi dua segmen.

\subsubsection{Algoritma}
\begin{algorithm}[H]
\caption{Crossover Satu Titik}
\begin{algorithmic}
\STATE $k \leftarrow$ RandomInteger(1, $l-1$)
\STATE $child1 \leftarrow parent1[1{:}k] \mathbin{+} parent2[k+1{:}l]$
\STATE $child2 \leftarrow parent2[1{:}k] \mathbin{+} parent1[k+1{:}l]$
\end{algorithmic}
\end{algorithm}

\subsubsection{Contoh}
\begin{align}
\text{Parent 1:} &\quad 1|1010011 \\
\text{Parent 2:} &\quad 0|0111100 \\
\text{Child 1:} &\quad 1|0111100 \\
\text{Child 2:} &\quad 0|1010011
\end{align}

Titik crossover di posisi 1.

\subsubsection{Karakteristik}

Crossover satu titik sederhana dan efisien; ia cenderung mempertahankan blok penyusun panjang di dekat ujung kromosom tetapi dapat mengganggu blok yang melintasi titik crossover, menghasilkan bias posisi di mana posisi ujung kurang mungkin terpisah.

\subsection{Crossover Dua Titik}

Dua titik crossover menghasilkan tiga segmen.

Crossover multi-titik memiliki mekanisme serupa dengan crossover satu titik; perbedaannya adalah jumlah posisi yang dipilih secara acak. Dalam crossover multi-titik, sejumlah N posisi dipilih secara acak sepanjang kromosom. Posisi-posisi tersebut ditukar untuk membentuk dua keturunan baru.

Tujuan dari crossover multi-titik antara lain meningkatkan keanekaragaman genetik dalam populasi, mengurangi bias posisi, dan meningkatkan peluang rekombinasi dari berbagai skema atau blok solusi yang berbeda. Namun, penggunaan terlalu banyak titik potong bisa menjadi terlalu merusak, karena dapat merusak kombinasi gen yang sudah baik (alel terkoadaptasi).

\begin{figure}[h]
\centering
\includegraphics[width=0.7\textwidth]{figures/buku_ajar_page_25.png}
\caption{Crossover Multi-Titik untuk kromosom biner}
\label{fig:binary_multi_point}
\end{figure}

\subsubsection{Algoritma}
\begin{algorithm}[H]
\caption{Crossover Dua Titik}
\begin{algorithmic}
\STATE $(k_1,k_2) \leftarrow$ Two distinct random integers with $1 \le k_1 < k_2 \le l-1$
\STATE $child1 \leftarrow parent1[1{:}k_1] \mathbin{+} parent2[k_1+1{:}k_2] \mathbin{+} parent1[k_2+1{:}l]$
\STATE $child2 \leftarrow parent2[1{:}k_1] \mathbin{+} parent1[k_1+1{:}k_2] \mathbin{+} parent2[k_2+1{:}l]$
\end{algorithmic}
\end{algorithm}

\subsubsection{Contoh}
\begin{align}
\text{Parent 1:} &\quad 11|010|011 \\
\text{Parent 2:} &\quad 00|111|100 \\
\text{Child 1:} &\quad 11|111|011 \\
\text{Child 2:} &\quad 00|010|100
\end{align}

Titik crossover di posisi 2 dan 5.

\subsubsection{Keuntungan}

Crossover dua titik mengurangi bias posisi dan dapat mempertahankan blok penyusun di ujung kromosom, meskipun secara umum lebih merusak dibandingkan crossover satu titik.

\subsection{Crossover Uniform}

Setiap gen dipilih secara independen dari salah satu induk~\cite{spears1993crossover, eshelman1991chc}.

Dalam crossover uniform, setiap gen (bit) dipilih secara acak dari salah satu gen yang sesuai pada kromosom induk. Proses pemilihan ini dilakukan secara independen untuk setiap posisi gen. Metode ini dapat dianalogikan seperti melempar koin: jika hasilnya "kepala", gen diambil dari Induk 1; jika hasilnya "ekor", gen diambil dari Induk 2.

Metode ini memberikan kesempatan yang sama bagi setiap gen untuk diwariskan dari salah satu induk, sehingga meningkatkan keanekaragaman genetik dan menghilangkan bias posisi yang biasanya muncul pada crossover satu titik atau multi-titik.

\begin{figure}[h]
\centering
\includegraphics[width=0.7\textwidth]{figures/buku_ajar_page_25.png}
\caption{Crossover Uniform untuk kromosom biner}
\label{fig:binary_uniform}
\end{figure}

\subsubsection{Algoritma}
\begin{algorithm}[H]
\caption{Crossover Uniform}
\begin{algorithmic}
\FOR{$i = 1$ to $l$}
    \STATE $r \leftarrow$ UniformRandom(0,1)
    \IF{$r < 0.5$}
        \STATE $child1[i] \leftarrow parent1[i]$; $child2[i] \leftarrow parent2[i]$
    \ELSE
        \STATE $child1[i] \leftarrow parent2[i]$; $child2[i] \leftarrow parent1[i]$
    \ENDIF
\ENDFOR
\end{algorithmic}
\end{algorithm}

\subsubsection{Contoh dengan Masker}
\begin{align}
\text{Parent 1:} &\quad 11010011 \\
\text{Parent 2:} &\quad 00111100 \\
\text{Mask:} &\quad 10110100 \\
\text{Child 1:} &\quad 10111011 \\
\text{Child 2:} &\quad 01010100
\end{align}

Bit masker 1: ambil dari Induk 1, bit masker 0: ambil dari Induk 2.

\subsubsection{Sifat}

Crossover uniform memiliki potensi disrupsi yang tinggi dan menghilangkan bias posisi; metode ini berguna ketika posisi gen bersifat independen namun dapat menghancurkan blok penyusun panjang.

\subsection{Crossover Multi-Titik}

Generalisasi dengan $k$ titik crossover.

\subsubsection{Karakteristik}

Crossover multi-titik merupakan generalisasi dari tanpa crossover ($k=0$, menyalin induk) melalui satu titik ($k=1$) hingga batas $k=l-1$ yang mendekati crossover uniform dalam harapan; peningkatan $k$ membuat operator mendekati rekombinasi uniform.

\section{Crossover untuk Kromosom Bilangan Bulat}

Berbeda dengan kromosom biner yang menggunakan bit 0 dan 1, representasi bilangan bulat lebih cocok untuk masalah yang melibatkan parameter diskret atau nilai numerik yang memiliki makna kuantitatif, seperti penjadwalan, pengurutan, atau optimasi kombinatorial.

\subsection{Crossover Satu Titik untuk Bilangan Bulat}

Crossover satu titik adalah bentuk paling sederhana dari crossover. Satu posisi crossover dipilih secara acak, lalu bagian dari dua induk setelah posisi tersebut ditukar untuk membentuk dua keturunan baru. String yang dihasilkan dari proses ini memiliki karakteristik bias posisi.

\begin{figure}[h]
\centering
\includegraphics[width=0.7\textwidth]{figures/buku_ajar_page_28.png}
\caption{Crossover Satu Titik untuk kromosom bilangan bulat}
\label{fig:integer_single_point}
\end{figure}

\subsection{Crossover Multi-Titik untuk Bilangan Bulat}

Crossover multi-titik memiliki mekanisme yang mirip dengan crossover satu titik; perbedaannya adalah jumlah posisi yang dipilih secara acak. Dalam crossover multi-titik, sejumlah N posisi dipilih secara acak sepanjang kromosom. Posisi-posisi ini ditukar untuk membentuk dua keturunan baru.

Tujuan dari crossover multi-titik termasuk meningkatkan keanekaragaman genetik dalam populasi, mengurangi bias posisi, dan meningkatkan peluang rekombinasi dari berbagai skema atau blok solusi yang berbeda. Namun, penggunaan terlalu banyak titik potong dapat merusak kombinasi gen yang sudah baik (alel terkoadaptasi).

\begin{figure}[h]
\centering
\includegraphics[width=0.7\textwidth]{figures/buku_ajar_page_28.png}
\caption{Crossover Multi-Titik untuk kromosom bilangan bulat}
\label{fig:integer_multi_point}
\end{figure}

\subsection{Crossover Uniform untuk Bilangan Bulat}

Dalam crossover uniform, setiap gen dipilih secara acak dari salah satu gen yang sesuai pada kromosom induk. Proses pemilihan ini dilakukan secara independen untuk setiap posisi gen. Metode ini bisa dianalogikan seperti melempar koin: jika hasilnya "kepala", gen diambil dari induk 1; jika hasilnya "ekor", gen diambil dari induk 2.

Metode ini memberikan kesempatan yang sama bagi setiap gen untuk diwariskan dari salah satu induk, sehingga meningkatkan keanekaragaman genetik dan menghilangkan bias posisi yang biasanya muncul pada crossover satu titik atau multi-titik.

\begin{figure}[h]
\centering
\includegraphics[width=0.7\textwidth]{figures/buku_ajar_page_29.png}
\caption{Crossover Uniform untuk kromosom bilangan bulat}
\label{fig:integer_uniform}
\end{figure}

\section{Operator Crossover untuk Kromosom Bernilai Riil}

Crossover pada kromosom bernilai riil adalah proses rekombinasi genetik dalam Algoritma Genetika yang diterapkan pada kromosom yang direpresentasikan dalam bentuk bilangan riil (floating point). Representasi ini umum digunakan untuk menyelesaikan masalah optimasi kontinu, di mana variabel keputusan memiliki nilai dalam rentang tertentu dan tidak dibatasi pada bilangan bulat atau biner.

Berbeda dari crossover pada kromosom biner atau bilangan bulat, mekanisme crossover untuk kromosom riil melibatkan operasi aritmatika pada nilai gen antara induk. Metode ini memungkinkan penciptaan keturunan dengan nilai gen yang berada di antara atau di sekitar nilai gen induk, sehingga mempertahankan kontinuitas dan stabilitas proses evolusi.

\subsection{Crossover Aritmetika}

Kombinasi linier vektor induk.

\subsubsection{Crossover Aritmetika Satu Titik}
\begin{enumerate}
    \item Dua induk direpresentasikan sebagai:
    \begin{itemize}
        \item Parent 1: $\langle x_1, \ldots, x_n \rangle$
        \item Parent 2: $\langle y_1, \ldots, y_n \rangle$
    \end{itemize}
    \item Pilih secara acak satu gen ($k$) yang akan mengalami operasi crossover
    \item Hasilnya adalah dua keturunan yang dibentuk berdasarkan kombinasi linier pada gen ke-$k$ dari induk dengan parameter kontrol $\alpha$, di mana $0 \leq \alpha \leq 1$:
    \begin{itemize}
        \item Offspring 1: $\langle x_1, \ldots, x_{k-1}, \alpha \cdot y_k + (1-\alpha) \cdot x_k, x_{k+1}, \ldots, x_n \rangle$
        \item Offspring 2: $\langle y_1, \ldots, y_{k-1}, \alpha \cdot x_k + (1-\alpha) \cdot y_k, y_{k+1}, \ldots, y_n \rangle$
    \end{itemize}
\end{enumerate}

\begin{figure}[h]
\centering
\includegraphics[width=0.7\textwidth]{figures/buku_ajar_page_31.png}
\caption{Crossover Aritmetika Satu Titik untuk kromosom riil}
\label{fig:real_single_arithmetic}
\end{figure}

\subsubsection{Crossover Aritmetika Sederhana}
\begin{enumerate}
    \item Dua induk direpresentasikan sebagai:
    \begin{itemize}
        \item Parent 1: $\langle x_1, \ldots, x_n \rangle$
        \item Parent 2: $\langle y_1, \ldots, y_n \rangle$
    \end{itemize}
    \item Pilih secara acak satu gen ($k$) sebagai titik batas crossover
    \item Hasilnya adalah dua keturunan yang dibentuk berdasarkan kombinasi linier dari gen $k+1$ sampai gen $n$ dengan parameter kontrol $\alpha$, di mana $0 \leq \alpha \leq 1$:
    \begin{itemize}
        \item Offspring 1: $\langle x_1, \ldots, x_k, \alpha \cdot y_{k+1} + (1-\alpha) \cdot x_{k+1}, \ldots, \alpha \cdot y_n + (1-\alpha) \cdot x_n \rangle$
        \item Offspring 2: $\langle y_1, \ldots, y_k, \alpha \cdot x_{k+1} + (1-\alpha) \cdot y_{k+1}, \ldots, \alpha \cdot x_n + (1-\alpha) \cdot y_n \rangle$
    \end{itemize}
\end{enumerate}

\begin{figure}[h]
\centering
\includegraphics[width=0.7\textwidth]{figures/buku_ajar_page_32.png}
\caption{Crossover Aritmetika Sederhana untuk kromosom riil}
\label{fig:real_simple_arithmetic}
\end{figure}

\subsubsection{Crossover Aritmetika Seluruhnya}
\begin{enumerate}
    \item Dua induk direpresentasikan sebagai:
    \begin{itemize}
        \item Parent 1: $\langle x_1, \ldots, x_n \rangle$
        \item Parent 2: $\langle y_1, \ldots, y_n \rangle$
    \end{itemize}
    \item Untuk setiap gen $i$ ($i = 1, 2, \, \ldots, n$), keturunan dibentuk dengan kombinasi linier gen dari kedua induk dengan parameter kontrol $\alpha$, di mana $0 \leq \alpha \leq 1$:
    \begin{itemize}
        \item Offspring 1: $z_i^1 = \alpha \cdot y_i + (1-\alpha) \cdot x_i$
        \item Offspring 2: $z_i^2 = \alpha \cdot x_i + (1-\alpha) \cdot y_i$
    \end{itemize}
\end{enumerate}

\begin{figure}[h]
\centering
\includegraphics[width=0.7\textwidth]{figures/buku_ajar_page_32.png}
\caption{Crossover Aritmetika Seluruhnya untuk kromosom riil}
\label{fig:real_whole_arithmetic}
\end{figure}

\subsubsection{Crossover Aritmetika Seluruhnya}
\begin{align}
\mathbf{child_1} &= \alpha \mathbf{parent_1} + (1-\alpha) \mathbf{parent_2} \\
\mathbf{child_2} &= (1-\alpha) \mathbf{parent_1} + \alpha \mathbf{parent_2}
\end{align}

di mana $\alpha \in [0,1]$ adalah bobot acak.

\subsubsection{Crossover Aritmetika Sederhana}

Terapkan crossover aritmetika pada subset gen yang dipilih secara acak.

\subsubsection{Crossover Aritmetika Satu Titik}

Terapkan crossover aritmetika pada satu gen yang dipilih secara acak.

\subsubsection{Contoh}
\begin{align}
\text{Parent 1:} &\quad (2.1, 5.7, 1.3, 8.9) \\
\text{Parent 2:} &\quad (4.2, 3.1, 6.8, 2.4) \\
\text{Child 1 ($\alpha=0.3$):} &\quad (3.57, 4.49, 4.98, 4.17) \\
\text{Child 2 ($\alpha=0.3$):} &\quad (2.73, 4.32, 2.98, 6.17)
\end{align}

\subsection{BLX-$\alpha$ Crossover (Blend Crossover)}

Menciptakan keturunan dalam interval di sekitar nilai induk.

\subsubsection{Algoritma}
\begin{algorithm}[H]
\caption{BLX-$\alpha$ Crossover}
\begin{algorithmic}
\FOR{$i = 1$ to $n$}
    \STATE $c_{min} \leftarrow \min(parent1[i], parent2[i])$
    \STATE $c_{max} \leftarrow \max(parent1[i], parent2[i])$
    \STATE $I \leftarrow c_{max} - c_{min}$
    \STATE Sample $child[i]$ uniformly from $[c_{min} - \alpha I,\; c_{max} + \alpha I]$
\ENDFOR
\end{algorithmic}
\end{algorithm}

\subsubsection{Parameter}
\begin{itemize}
    \item $\alpha = 0$: Offspring between parents
    \item $\alpha = 0.5$: Standard BLX-0.5
    \item Larger $\alpha$: More exploration beyond parents
\end{itemize}

\subsection{SBX (Simulated Binary Crossover)}

Mensimulasikan perilaku crossover satu titik untuk gen bernilai riil.

\subsubsection{Rumus}
\begin{align}
child_{1i} &= 0.5[(1+\beta_i)parent_{1i} + (1-\beta_i)parent_{2i}] \\
child_{2i} &= 0.5[(1-\beta_i)parent_{1i} + (1+\beta_i)parent_{2i}]
\end{align}

di mana $\beta_i$ dihitung dari:
\begin{equation}
\beta_i = \begin{cases}
(2u_i)^{1/(\eta_c+1)} & \text{if } u_i \leq 0.5 \\
\left(\frac{1}{2(1-u_i)}\right)^{1/(\eta_c+1)} & \text{if } u_i > 0.5
\end{cases}
\end{equation}

$u_i \sim U[0,1]$ dan $\eta_c$ adalah indeks distribusi.

\section{Operator Crossover untuk Permutasi}

\subsection{Order Crossover (OX)}

Mempertahankan urutan relatif elemen dari satu induk~\cite{oliver1987study, larranaga1999genetic}.

\subsubsection{Algoritma}
\begin{algorithm}[H]
\caption{Order Crossover (OX)}
\begin{algorithmic}
\STATE Choose two crossover points $a,b$ with $1 \le a < b \le n$
\STATE Copy segment $parent1[a{:}b]$ into $child[a{:}b]$
\STATE $pos \leftarrow b+1$ (wrap to 1 if $> n$)
\FOR{each element $x$ in $parent2$ in order}
    \IF{$x$ not in $child$}
        \STATE $child[pos] \leftarrow x$
        \STATE $pos \leftarrow pos+1$ (wrap to 1 if $> n$)
    \ENDIF
\ENDFOR
\end{algorithmic}
\end{algorithm}

\subsubsection{Contoh}
\begin{align}
\text{Parent 1:} &\quad (1, 2, 3, 4, 5, 6, 7, 8, 9) \\
\text{Parent 2:} &\quad (9, 3, 7, 8, 2, 6, 5, 1, 4) \\
\text{Copy segment:} &\quad (\_, \, \_, 3, 4, 5, 6, \, \_, \, \_) \\
\text{Fill from P2:} &\quad (7, 8, 3, 4, 5, 6, 2, 1, 9)
\end{align}

\subsection{Partially Mapped Crossover (PMX)}

Menciptakan pemetaan antar elemen di dalam segmen crossover~\cite{goldberg1989genetic, larranaga1999genetic}.

\subsubsection{Algoritma}
\begin{algorithm}[H]
\caption{Partially Mapped Crossover (PMX)}
\begin{algorithmic}
\STATE Choose two crossover points $a,b$ with $1 \le a < b \le n$
\STATE Copy $parent1[a{:}b]$ into $child1[a{:}b]$ and $parent2[a{:}b]$ into $child2[a{:}b]$
\STATE Create mapping pairs from exchanged segments
\FOR{each position $i$ outside $[a,b]$}
    \STATE $val \leftarrow parent1[i]$
    \WHILE{$val$ is already in $child1[a{:}b]$}
        \STATE $val \leftarrow$ mapping of $val$ (follow mapping until an unused value found)
    \ENDWHILE
    \STATE $child1[i] \leftarrow val$
    \COMMENT{Repeat symmetrically for $child2$}
\ENDFOR
\end{algorithmic}
\end{algorithm}

\subsubsection{Contoh}
\begin{align}
\text{Parent 1:} &\quad (1, 2, 3, 4, 5, 6, 7, 8, 9) \\
\text{Parent 2:} &\quad (5, 4, 6, 9, 2, 3, 7, 1, 8) \\
\text{Mapping:} &\quad 3 \leftrightarrow 6, 4 \leftrightarrow 9, 5 \leftrightarrow 2, 6 \leftrightarrow 3 \\
\text{Child 1:} &\quad (1, 5, 6, 9, 2, 3, 7, 8, 4)
\end{align}

\subsection{Cycle Crossover (CX)}

Mempertahankan posisi absolut elemen dari kedua induk~\cite{oliver1987study, reeves1993modern}.

\subsubsection{Algoritma}
\begin{algorithm}[H]
\caption{Cycle Crossover (CX)}
\begin{algorithmic}
\STATE Initialize all positions as unassigned
\STATE $cycleStart \leftarrow 1$
\WHILE{there are unassigned positions}
    \STATE $i \leftarrow cycleStart$
    \STATE Build cycle: add $i$ to cycle; set $i \leftarrow$ position of $parent1[i]$ in $parent2$; repeat until returning to $cycleStart$
    \STATE For indices in cycle assign values from $parent1$ to $child1$ and from $parent2$ to $child2$
    \STATE Choose next unassigned position as new $cycleStart$
\ENDWHILE
\end{algorithmic}
\end{algorithm}

\subsection{Edge Recombination Crossover}

Mempertahankan informasi tepi dari kedua induk (berguna untuk TSP).

\subsubsection{Algoritma}
\begin{algorithm}[H]
\caption{Edge Recombination Crossover}
\begin{algorithmic}
\STATE Build edge table: for each element list its neighbors from both parents (no duplicates)
\STATE $current \leftarrow$ element with fewest edges (break ties randomly)
\FOR{$pos = 1$ to $n$}
    \STATE $child[pos] \leftarrow current$
    \STATE Remove $current$ from all edge lists
    \IF{edge table of $current$ has any neighbors}
        \STATE $next \leftarrow$ neighbor of $current$ with fewest edges (break ties randomly)
    \ELSE
        \STATE $next \leftarrow$ random unused element
    \ENDIF
    \STATE $current \leftarrow next$
\ENDFOR
\end{algorithmic}
\end{algorithm}

\section{Analisis Crossover}

\subsection{Disrupsi Skema}

Probabilitas bahwa sebuah skema $H$ terganggu oleh crossover:

\subsubsection{One-Point Crossover}
\begin{equation}
P_{disruption} = p_c \cdot \frac{\delta(H)}{l-1}
\end{equation}

\subsubsection{Two-Point Crossover}
\begin{equation}
P_{disruption} = p_c \cdot \left(\frac{2\delta(H)}{l-1} - \frac{\delta(H)(\delta(H)-1)}{(l-1)(l-2)}\right)
\end{equation}

\subsubsection{Uniform Crossover}
\begin{equation}
P_{disruption} = p_c \cdot \left(1 - \left(\frac{1}{2}\right)^{o(H)-1}\right)
\end{equation}

\subsection{Pelestarian Blok Penyusun}

Skema pendek umumnya lebih baik dilestarikan oleh operator crossover, sedangkan skema panjang lebih rentan terganggu (khususnya oleh crossover uniform); gen yang terikat erat mendapat manfaat dari operator permutasi seperti order crossover yang mempertahankan hubungan kedekatan.

\section{Teknik Crossover Lanjutan}

\subsection{Crossover Adaptif}

Crossover adaptif menyesuaikan parameter secara dinamis berdasarkan sinyal seperti keanekaragaman populasi, laju peningkatan fitness, jumlah generasi, dan tingkat fitness individu.

\subsection{Crossover Multi-Induk}

Crossover multi-induk menggabungkan materi dari lebih dari dua induk (lihat \cite{elsayed2011ga, yassen2012multi, fajrin2020multi, chen2013multiple}); contoh-contohnya termasuk scanning crossovers yang menjelajah induk secara berurutan, voting crossovers yang menggunakan aturan mayoritas, dan averaging crossovers yang menghitung rata-rata nilai gen induk.

\subsection{Crossover Khusus Masalah}

Operator crossover khusus masalah dirancang untuk mempertahankan batasan domain dan struktur berguna: untuk masalah graf mempertahankan properti graf, untuk penjadwalan mempertahankan batasan temporal, dan untuk jaringan saraf mempertahankan topologi jaringan.

\section{Panduan Crossover}

\subsection{Memilih Jenis Crossover}

Pilih jenis crossover sesuai representasi: untuk biner gunakan satu titik, dua titik, atau uniform; untuk bernilai riil gunakan aritmetika, BLX-$\alpha$, atau SBX; untuk permutasi gunakan OX, PMX, atau CX tergantung struktur masalah; representasi dengan panjang variabel membutuhkan operator khusus.

\subsection{Penetapan Parameter}

Saran parameter tipikal: mulai dengan laju crossover $p_c\approx0.8$--$0.9$; populasi yang lebih besar dapat mentolerir laju crossover lebih tinggi; masalah yang lebih sulit mungkin mendapat manfaat dari laju yang lebih rendah dan rekombinasi yang lebih konservatif.

\subsection{Pengujian Empiris}

Pengujian empiris harus membandingkan beberapa operator crossover, memvariasikan parameter, mengukur keragaman dan konvergensi, serta memasukkan metrik khusus masalah untuk memvalidasi pilihan.
