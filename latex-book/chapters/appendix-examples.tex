% Appendix: Contoh Praktis dan Studi Kasus (terjemahan bahasa Indonesia)
\chapter{Contoh Praktis dan Studi Kasus}

\section{Masalah Optimasi Fungsi}

\subsection{OneMax}
OneMax adalah contoh sederhana untuk algoritma genetika biner. Tujuan: maksimalkan jumlah bit 1 dalam kromosom.
\begin{equation}
f(x)=\sum_{i=1}^n x_i
\end{equation}

\subsection{Beberapa Fungsi Benchmark}
\begin{itemize}
    \item \textbf{Sphere}: $f(\mathbf{x})=\sum_{i=1}^n x_i^2$, domain $[-5.12,5.12]^n$, minimum global di $\mathbf{0}$.
    \item \textbf{Rastrigin}: multimodal, $f(\mathbf{x})=An+\sum[x_i^2-A\cos(2\pi x_i)]$, $A=10$.
    \item \textbf{Rosenbrock}: lembah sempit, $f(\mathbf{x})=\sum[100(x_{i+1}-x_i^2)^2+(1-x_i)^2]$.
\end{itemize}

\section{Masalah Optimasi Kombinatorial}

\subsection{TSP}
Traveling Salesman Problem: temukan rute terpendek yang melewati semua kota sekali dan kembali ke awal. Pada GA digunakan encoding permutasi dan operator khusus (PMX, OX), serta langkah perbaikan lokal.

\subsection{Knapsack 0/1}
Pilih item untuk memaksimalkan nilai dengan batas bobot $W$:
\begin{align}
\max\; &\sum_{i=1}^n v_i x_i \\
	ext{s.t. } &\sum_{i=1}^n w_i x_i \le W, \quad x_i\in\{0,1\}.
\end{align}

\section{Aplikasi Dunia Nyata}

\subsection{Pelatihan Jaringan Syaraf}
GA dapat mengoptimalkan bobot atau arsitektur jaringan dengan encoding vektor real; contoh metrik: $fitness=1/(1+MSE)$.

\subsection{Seleksi Fitur}
Encoding biner memungkinkan GA memilih subset fitur. Tujuan sering kali multi-objektif: maksimalkan performa sambil minimalkan jumlah fitur.

\subsection{Penjadwalan}
Job shop scheduling dapat direpresentasikan dengan permutasi atau prioritas; GA sering digabungkan dengan pencarian lokal (2-opt, 3-opt).

\section{Panduan Penyetelan Singkat}
\begin{itemize}
    \item Ukuran populasi: sederhana $50$--$100$, sedang $100$--$500$, kompleks $500$--$2000$.
    \item Crossover: umum $0.7$--$0.9$.
    \item Mutasi: binary ~$1/L$, real $0.01$--$0.1$.
    \item Seleksi: turnamen (size 2--7) untuk mengatur tekanan seleksi.
\end{itemize}

\section{Analisis Performa}
Gunakan metrik: best/average fitness, keberagaman, success rate. Jalankan 20--30 trial independen, laporkan mean, std, best, worst, dan lakukan uji statistik bila perlu.

\section{Kendala Umum dan Solusi}
\begin{itemize}
    \item Konvergensi dini: tingkatkan populasi, kurangi tekanan seleksi, tingkatkan mutasi, gunakan mekanisme pelestarian keberagaman.
    \item Konvergensi lambat: tingkatkan tekanan seleksi, tambahkan pencarian lokal.
    \item Kendala: gunakan penalti adaptif, repair, atau pendekatan multi-objektif.
\end{itemize}

\section{Teknik Lanjutan}
Hibridisasi (memetic algorithms), kontrol parameter adaptif/self-adaptive, dan paralelisasi (master-slave, island model, cellular GA) sering meningkatkan performa.

\section{Ringkasan}
Ringkasan singkat: desain representasi, tuning parameter, validasi statistik, dan penggunaan teknik hibrida adalah kunci keberhasilan penerapan GA.

% Akhir file
    \begin{itemize}
        \item Sedikit perbaikan selama banyak generasi
        \item Keberagaman populasi tetap tinggi
        \item Perilaku seperti random walk
    \end{itemize}

    	extbf{Solusi:}
    \begin{itemize}
        \item Tingkatkan tekanan seleksi
        \item Kurangi laju mutasi
        \item Terapkan pencarian lokal (GA hibrida)
        \item Gunakan inisialisasi yang lebih baik
        \item Sesuaikan operator crossover
    \end{itemize}

    \subsection{Masalah Penanganan Kendala}
    	extbf{Masalah Umum:}
    \begin{itemize}
        \item Semua individu melanggar kendala
        \item Wilayah feasibel terlalu kecil
        \item Koefisien penalti disetel buruk
    \end{itemize}

    	extbf{Solusi:}
    \begin{itemize}
        \item Gunakan mekanisme perbaikan
        \item Terapkan operator khusus
        \item Implementasikan pelestarian feasibilitas
        \item Gunakan pendekatan multi-objektif
        \item Sesuaikan bobot penalti secara dinamis
    \end{itemize}

    \section{Teknik Lanjutan}

    \subsection{Genetic Algorithms Hibrida}
    Gabungkan GA dengan metode pencarian lokal:
    \begin{itemize}
        \item \textbf{Memetic algorithms}: GA + pencarian lokal
        \item \textbf{Evolusi Lamarckian}: Mewariskan solusi yang telah diperbaiki
        \item \textbf{Evolusi Baldwinian}: Gunakan pencarian lokal hanya untuk evaluasi fitness
    \end{itemize}

    \subsection{Kontrol Parameter Adaptif}
    Sesuaikan parameter GA secara otomatis selama evolusi:
    \begin{itemize}
        \item \textbf{Deterministik}: Jadwal pra-definisi
        \item \textbf{Adaptif}: Berdasarkan keadaan populasi
        \item \textbf{Self-adaptive}: Parameter berevolusi bersama populasi
    \end{itemize}

    \subsection{Parallel Genetic Algorithms}
    Distribusikan komputasi ke beberapa prosesor:
    \begin{itemize}
        \item \textbf{Master-slave}: Evaluasi fitness paralel
        \item \textbf{Island model}: Beberapa populasi dengan migrasi
        \item \textbf{Cellular GA}: Struktur populasi spasial
    \end{itemize}

    \section{Praktik Terbaik Implementasi}

    \subsection{Organisasi Kode}
    \begin{itemize}
        \item Pisahkan representasi dari operator
        \item Gunakan desain modular untuk kemudahan pengujian
        \item Implementasikan generator bilangan acak yang baik
        \item Tambahkan logging dan kemampuan visualisasi
    \end{itemize}

    \subsection{Pengujian dan Validasi}
    \begin{itemize}
        \item Uji pada masalah benchmark yang dikenal
        \item Verifikasi operator menjaga validitas solusi
        \item Periksa kualitas generator bilangan acak
        \item Profil dulu hambatan performa
    \end{itemize}

    \subsection{Dokumentasi}
    \begin{itemize}
        \item Dokumentasikan pilihan parameter dan alasannya
        \item Catat detail pengaturan eksperimen
        \item Pertahankan kontrol versi
        \item Bagikan hasil yang dapat direproduksi
    \end{itemize}

    \section{Ringkasan Bab}

    Bab ini menyajikan contoh praktis dan studi kasus yang menunjukkan penerapan algoritma genetika pada berbagai domain masalah. Pelajaran penting meliputi pentingnya desain representasi yang tepat, penyetelan parameter, dan analisis performa. Memahami kendala umum dan solusinya sangat krusial untuk implementasi GA yang berhasil.

    \section{Poin Penting}
    \begin{itemize}
        \item Representasi masalah sangat menentukan keberhasilan GA
        \item Pengaturan parameter harus sesuai karakteristik masalah
        \item Validasi statistik memastikan hasil yang dapat dipercaya
        \item Pendekatan hibrida seringkali mengungguli GA murni
        \item Pengetahuan domain harus memandu desain operator
        \item Pengujian dan dokumentasi yang baik adalah esensial
    \end{itemize}