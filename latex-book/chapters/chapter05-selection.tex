\chapter{Metode Seleksi dalam Algoritma Genetika}

\section{Pendahuluan tentang Seleksi}

Seleksi adalah mekanisme dalam algoritma genetika yang menentukan individu mana dari populasi yang dipilih untuk menyumbangkan materi genetik ke generasi berikutnya. Pada inti operasinya, seleksi mengubah informasi kebugaran (fitness) menjadi peluang reproduksi: individu dengan nilai kebugaran relatif lebih tinggi diberi peluang lebih besar untuk menghasilkan keturunan, sehingga memfokuskan pencarian ke wilayah solusi yang menjanjikan. Bias ini harus dikelola dengan hati‑hati agar algoritma dapat mengeksploitasi solusi berkualitas tinggi sekaligus tetap mengeksplorasi alternatif yang beragam.

Konsep penting yang terkait dengan seleksi adalah tekanan seleksi (selection pressure), yang mengukur seberapa kuat mekanisme seleksi memfavoritkan individu yang lebih baik. Tekanan seleksi tinggi mempercepat konvergensi dengan memperbesar keuntungan reproduktif individu terbaik, namun meningkatkan risiko konvergensi prematur saat populasi kehilangan keragaman dan terjebak pada solusi suboptimal. Tekanan seleksi rendah mempertahankan keragaman dan mendorong eksplorasi, tetapi dapat memperlambat kemajuan menuju solusi berkualitas. Oleh karena itu, desain praktis algoritma membutuhkan penyeimbangan efek‑efek ini, misalnya dengan pengaturan parameter seleksi atau dengan menggabungkan skema seleksi bersama mekanisme yang mempertahankan keragaman.

Operator seleksi terbagi ke dalam beberapa keluarga yang memberikan kompromi berbeda antara kesederhanaan, kontrol tekanan seleksi, dan sensitivitas terhadap skala kebugaran. Pendekatan umum meliputi metode proporsional terhadap kebugaran (mis. roulette wheel dan stochastic universal sampling), skema berbasis peringkat yang memberikan tekanan terkontrol dan independen skala, seleksi turnamen yang efisien dan dapat diatur tekanannya, serta strategi tronkasi atau elitisme yang secara deterministik mempertahankan individu terbaik. Bagian‑bagian selanjutnya menguraikan metode tersebut secara detail, termasuk algoritma, sifat statistik, serta kelebihan dan kekurangan praktisnya.

\begin{figure}[h]
\centering
\includegraphics[width=0.75\textwidth]{figures/buku_ajar_page_16.png}
\caption{Proses seleksi dasar dalam Algoritma Genetika}
\label{fig:selection_basic_process}
\end{figure}

\section{Tekanan Seleksi}
Tekanan seleksi mengkuantifikasi seberapa kuat suatu mekanisme seleksi memfavoritkan individu dengan kebugaran lebih tinggi saat menghasilkan generasi berikutnya. Secara intuitif, ia mengukur keuntungan reproduktif yang diharapkan dari solusi baik relatif terhadap rata‑rata populasi. Tekanan seleksi dapat diformalkan dengan berbagai cara; ukuran operasional yang umum antara lain intensitas seleksi (perbedaan terstandarisasi antara rata‑rata orang tua dan populasi) dan waktu takeover (jumlah generasi yang dibutuhkan agar individu terbaik mendominasi populasi bila seleksi diulang). Ukuran‑ukuran ini memungkinkan perbandingan terukur antara operator seleksi dan parametrisasinya.

Pengendalian praktis terhadap tekanan seleksi meliputi pilihan algoritmik (mis. ukuran turnamen, kemiringan peringkat, fraksi tronkasi), teknik penskalaan kebugaran (mis. linear atau sigma scaling, seleksi Boltzmann), serta strategi hibrida yang menyesuaikan tekanan selama run (mis. mulai dengan tekanan rendah untuk eksplorasi lalu tingkatkan untuk eksploitasi). Memantau statistik terkait seleksi — seperti rata‑rata dan varians kebugaran, ukuran keragaman (mis. jarak Hamming rata‑rata pada enkoding biner), dan perkiraan takeover time — memberi umpan balik berguna untuk tuning.

\section{Seleksi Proporsional terhadap Kebugaran (FPS)}

Algoritma Genetika yang dikembangkan oleh Holland menggunakan Fitness Proportionate Selection (FPS)~\cite{holland1975adaptation, goldberg1989genetic}, dimana nilai harapan suatu individu dihitung sebagai rasio kebugaran individu terhadap kebugaran rata‑rata populasi.

Dalam metode ini, setiap individu dipilih sebagai orang tua dengan probabilitas proporsional terhadap nilai kebugarannya. Dengan demikian, individu yang lebih fit memiliki peluang lebih besar untuk mereproduksi dan menyebarkan sifatnya ke generasi berikutnya.

\subsection{Seleksi Roulette Wheel}
Dikenal juga sebagai seleksi proporsional kebugaran, di mana individu dipilih dengan probabilitas proporsional terhadap kebugarannya~\cite{goldberg1989genetic, obitko_selection, algorithmafternoon_selection}.

Skema seleksi paling sederhana adalah roulette wheel, atau pengambilan sampel stokastik dengan penggantian. Individu dipetakan ke segmen‑segmen pada garis sesuai nilai kebugaran; bilangan acak kemudian menentukan segmen (individu) yang terpilih. Proses diulang hingga jumlah individu yang diinginkan terpenuhi.

\begin{table}[H]
\centering
\begin{tabular}{cccc}
\toprule
Nomor Individu & Nilai Kebugaran & Probabilitas Seleksi & Interval \\
\midrule
1 & 2.0 & 0.18 & [0.00, 0.18] \\
2 & 1.8 & 0.16 & [0.18, 0.34] \\
3 & 1.6 & 0.15 & [0.34, 0.49] \\
4 & 1.4 & 0.13 & [0.49, 0.62] \\
5 & 1.2 & 0.11 & [0.62, 0.73] \\
6 & 1.0 & 0.09 & [0.73, 0.82] \\
7 & 0.8 & 0.07 & [0.82, 0.89] \\
8 & 0.6 & 0.06 & [0.89, 0.95] \\
9 & 0.4 & 0.03 & [0.95, 0.98] \\
10 & 0.2 & 0.02 & [0.98, 1.00] \\
11 & 0.0 & 0.0 & -- \\
\bottomrule
\end{tabular}
\caption{Probabilitas seleksi dan nilai kebugaran (dari Buku Ajar)}
\label{tab:selection_probability}
\end{table}

\begin{figure}[h]
\centering
\includegraphics[width=0.85\textwidth]{figures/buku_ajar_page_18.png}
\caption{Proses seleksi roulette‑wheel dengan contoh pengundian}
\label{fig:roulette_wheel_selection}
\end{figure}

\subsubsection{Algoritma}
\begin{algorithm}
\caption{Roulette Wheel Selection}
\begin{algorithmic}
\STATE Hitung total kebugaran: $F = \sum_{i=1}^{N} f_i$
\STATE Hasilkan bilangan acak: $r \sim U[0, F]$
\STATE Set kumulatif kebugaran: $sum = 0$
\FOR{$i = 1$ to $N$}
    \STATE $sum = sum + f_i$
    \IF{$sum \geq r$}
        \STATE Pilih individu $i$
        \STATE \textbf{break}
    \ENDIF
\ENDFOR
\end{algorithmic}
\end{algorithm}

\subsubsection{Probabilitas Seleksi}
\begin{equation}
P_i = \frac{f_i}{\sum_{j=1}^{N} f_j}
\end{equation}

\subsection{Stochastic Universal Sampling (SUS)}
Versi yang ditingkatkan dari roulette‑wheel yang mengurangi variansi sampling~\cite{baker1987reducing}.

\begin{figure}[h]
\centering
\includegraphics[width=0.85\textwidth]{figures/buku_ajar_page_19.png}
\caption{Stochastic universal sampling dengan penunjuk berjarak sama}
\label{fig:sus_selection}
\end{figure}

\subsubsection{Algoritma}
\begin{algorithm}
\caption{Stochastic Universal Sampling}
\begin{algorithmic}
\STATE Hitung total kebugaran: $F = \sum_{i=1}^{N} f_i$
\STATE Hitung jarak pointer: $distance = F / N$
\STATE Hasilkan start acak: $start \sim U[0, distance]$
\STATE Buat pointer: $pointer_i = start + i \times distance$ untuk $i = 0, 1, \ldots, N-1$
\FOR{setiap pointer}
    \STATE Pilih individu menggunakan logika roulette wheel
\ENDFOR
\end{algorithmic}
\end{algorithm}

\section{Seleksi Berbasis Peringkat}

Seleksi berbasis peringkat menetapkan probabilitas seleksi berdasarkan peringkat kebugaran, bukan nilai kebugaran mentah~\cite{grefenstette1986optimization, algorithmafternoon_ranked}.

\subsection{Perankingan Linear}
\begin{equation}
P_i = \frac{1}{N} \left[ \eta^- + (\eta^+ - \eta^-) \frac{rank_i - 1}{N - 1} \right]
\end{equation}

di mana $rank_i$ adalah peringkat individu $i$ (1 = terburuk, $N$ = terbaik) dan $\eta^+ + \eta^- = 2$.

\section{Seleksi Turnamen}

Seleksi turnamen memilih secara acak $k$ individu dan memilih yang terbaik di antara mereka~\cite{goldberg1989genetic}.

\subsection{Mekanisme}
\begin{enumerate}
    \item Tentukan ukuran turnamen ($k$).
    \item Pilih secara acak $k$ individu.
    \item Pilih individu dengan kebugaran tertinggi di antara mereka.
    \item Tambahkan pemenang ke mating pool.
    \item Ulangi hingga jumlah yang diinginkan tercapai.
\end{enumerate}

\begin{algorithm}
\caption{Binary Tournament Selection}
\begin{algorithmic}
\STATE Pilih acak individu $i$
\STATE Pilih acak individu $j$ (dengan $j \neq i$)
\IF{$f_i > f_j$}
    \STATE Pilih individu $i$
\ELSE
    \STATE Pilih individu $j$
\ENDIF
\end{algorithmic}
\end{algorithm}

\section{Truncation Selection}

Truncation selection mempertahankan hanya fraksi teratas populasi untuk reproduksi. Parameter utama adalah rasio seleksi
\begin{equation}
\rho = \frac{\mu}{\lambda},
\end{equation}
yang mengontrol tekanan seleksi: $\rho$ kecil berarti tekanan kuat.

\section{Boltzmann Selection}

Seleksi Boltzmann memetakan kebugaran ke probabilitas menggunakan distribusi Gibbs:
\begin{equation}
P_i = \frac{e^{f_i/T}}{\sum_{j=1}^{N} e^{f_j/T}},
\end{equation}
di mana $T$ adalah parameter temperatur yang dapat dijadwalkan selama run.

\section{Elitist Selection}

Elitisme menjamin kelangsungan hidup sejumlah kecil individu terbaik antar generasi (mis. $e=1$).

\section{Seleksi yang Mempertahankan Keragaman}

Teknik seperti fitness sharing, crowding, spesiasi, dan model pulau bertujuan mempertahankan variasi genetik. Contoh rumus fitness sharing:
\begin{equation}
f'_i = \frac{f_i}{\sum_{j=1}^{N} sh(d_{ij})},
\end{equation}
dengan fungsi sharing tipikal:
\begin{equation}
sh(d) = \begin{cases}
1 - \left(\dfrac{d}{\sigma_{share}}\right)^\alpha & \text{jika } d < \sigma_{share}, \\
0 & \text{lainnya.}
\end{cases}
\end{equation}

\section{Seleksi Multi‑objektif}

Untuk masalah multi‑objektif, dominasi Pareto dan non‑dominated sorting (mis. NSGA‑II) digunakan untuk menyeimbangkan konvergensi dan penyebaran solusi.

\section{Perbandingan Metode Seleksi}
\begin{table}[H]
\centering
\scriptsize
\begin{tabular}{lccccc}
\toprule
Metode & Tekanan & Keragaman & Kompleksitas & Skalabilitas & Parameter \\
\midrule
Roulette Wheel & Variabel & Lemah & $O(N)$ & Lemah & --- \\
SUS & Variabel & Baik & $O(N)$ & Lemah & --- \\
Rank Linear & Konstan & Baik & $O(N \log N)$ & Baik & $\eta^+, \eta^-$ \\
Tournament & Dapat Diatur & Baik & $O(1)$ & Sangat Baik & $k$ \\
Truncation & Tinggi & Lemah & $O(N \log N)$ & Baik & $\mu/\lambda$ \\
Boltzmann & Adaptif & Sangat Baik & $O(N)$ & Baik & $T(t)$ \\
\bottomrule
\end{tabular}
\caption{Perbandingan Metode Seleksi}
\end{table}
