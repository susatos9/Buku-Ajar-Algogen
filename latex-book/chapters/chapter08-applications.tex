\chapter{Real-World Applications and Visual Examples}
\label{ch:applications}




This chapter showcases real-world applications of genetic algorithms drawn from the
course materials and demonstrates how GA concepts are applied in practice~\cite{gen2007genetic,
sivanandam2008introduction, eiben2015introduction}.

One of the most compelling demonstrations is the application of genetic algorithms to
game AI. The course contains an example of a GA-based agent that beats the first level
of Super Mario Bros. at \(4\times\) speed \cite{montana1989training, koza1992genetic}.

The "Towers of Reus" project demonstrates how a GA can be used for gameplay
balancing. Users create maps with adjustable parameters while a GA searches for
configurations that produce desirable win/lose characteristics. The system can then
report whether towers are too strong or too weak and present beatable levels for
players to test.

Another example in the course is path-finding: the problem is to find the shortest
path through a complex maze. The encoding is a sequence of movement directions
(up, down, left, right)~\cite{larranaga1999genetic}. A suitable fitness function is the
inverse of the path length with penalties for hitting walls. Crossover is used to
combine successful path segments and mutation explores new moves.

Physical robot navigation shows how GAs transfer to hardware applications. Use
cases include real-time path planning in dynamic environments, integrating sensor
data for obstacle avoidance, and evolving adaptive behaviors based on environmental
feedback.

The course also references simulated-evolution examples available online at
\url{http://www.wreck.devisland.net/ga/}. These examples illustrate features such as
morphology evolution (changes to body structure), locomotion pattern optimization,
environmental adaptation, and multi-objective fitness criteria like speed,
stability, and efficiency~\cite{deb2001multi, deb2002fast, horn1994niched}.

An intuitive analogy used in the materials compares a population of individuals
with varying physical abilities: selection favors those who jump higher, inheritance
passes traits to subsequent generations, and mutation introduces random technique
variations.

A practical optimisation example is daily-commute planning. In this analogy, route
options act as genes, traffic conditions act as environmental factors, and time and
fuel consumption serve as fitness criteria. Over repeated generations the system can
learn to prefer more efficient routes.

The chapter also highlights GA's relationship with other paradigms: machine
learning (kernel methods, SVMs, Hidden Markov models, Bayesian methods), soft
computing (neural networks and fuzzy systems), and hybrid approaches such as
reinforcement learning.

From a conceptual standpoint the course contrasts Lamarck's idea of acquired
characteristics with the Darwin-Wallace view of selection. Genetic algorithms
follow Darwinian principles by applying random variation and selection to search
for good solutions. These visual examples demonstrate GA's wide applicability across entertainment
(game AI and procedural content generation), robotics (path planning and adaptive
behaviour), simulation (artificial life and evolution studies), optimization (route
planning and resource allocation), and research (studying evolutionary processes).
The key insight is that GAs provide a unified framework for solving complex
optimization problems across diverse domains, making them a versatile tool in
computational intelligence.
