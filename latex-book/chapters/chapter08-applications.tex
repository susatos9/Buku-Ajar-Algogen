\chapter{Aplikasi Algoritma Genetika}
\label{ch:applications}




Bab ini menampilkan aplikasi dunia nyata dari algoritma genetika yang diambil dari
materi perkuliahan dan mendemonstrasikan bagaimana konsep GA diterapkan dalam praktik~\cite{gen2007genetic,
sivanandam2008introduction, eiben2015introduction}.

Salah satu demonstrasi yang paling menarik adalah penerapan algoritma genetika pada
AI game. Perkuliahan ini memuat contoh agen berbasis GA yang mengalahkan level pertama
Super Mario Bros. dengan kecepatan \(4\times\) \cite{montana1989training, koza1992genetic}.

Proyek "Towers of Reus" mendemonstrasikan bagaimana GA dapat digunakan untuk
penyeimbangan gameplay. Pengguna membuat peta dengan parameter yang dapat disesuaikan sementara GA mencari
konfigurasi yang menghasilkan karakteristik menang/kalah yang diinginkan. Sistem kemudian dapat
melaporkan apakah menara terlalu kuat atau terlalu lemah dan menyajikan level yang dapat dikalahkan untuk
diuji oleh pemain.

Contoh lain dalam perkuliahan adalah pencarian jalur: masalahnya adalah menemukan jalur
terpendek melalui labirin yang kompleks. Pengkodean berupa urutan arah gerakan
(atas, bawah, kiri, kanan)~\cite{larranaga1999genetic}. Fungsi fitness yang sesuai adalah
kebalikan dari panjang jalur dengan penalti untuk menabrak dinding. Crossover digunakan untuk
menggabungkan segmen jalur yang berhasil dan mutasi mengeksplorasi gerakan baru.

Navigasi robot fisik menunjukkan bagaimana GA ditransfer ke aplikasi perangkat keras. Kasus
penggunaan meliputi perencanaan jalur real-time dalam lingkungan dinamis, mengintegrasikan data
sensor untuk penghindaran rintangan, dan mengevolusi perilaku adaptif berdasarkan umpan balik
lingkungan.

Perkuliahan ini juga merujuk contoh-contoh evolusi tersimulasi yang tersedia online di
\url{http://www.wreck.devisland.net/ga/}. Contoh-contoh ini mengilustrasikan fitur seperti
evolusi morfologi (perubahan pada struktur tubuh), optimasi pola pergerakan,
adaptasi lingkungan, dan kriteria fitness multi-objektif seperti kecepatan,
stabilitas, dan efisiensi~\cite{deb2001multi, deb2002fast, horn1994niched}.

Analogi intuitif yang digunakan dalam materi membandingkan populasi individu
dengan kemampuan fisik yang bervariasi: seleksi menguntungkan mereka yang melompat lebih tinggi, pewarisan
menurunkan sifat ke generasi berikutnya, dan mutasi memperkenalkan variasi teknik
acak.

Contoh optimasi praktis adalah perencanaan perjalanan harian. Dalam analogi ini, pilihan
rute bertindak sebagai gen, kondisi lalu lintas bertindak sebagai faktor lingkungan, dan waktu serta
konsumsi bahan bakar berfungsi sebagai kriteria fitness. Selama generasi berulang sistem dapat
belajar untuk memilih rute yang lebih efisien.

Bab ini juga menyoroti hubungan GA dengan paradigma lain: pembelajaran
mesin (metode kernel, SVM, model Hidden Markov, metode Bayesian), komputasi
lunak (jaringan saraf dan sistem fuzzy), dan pendekatan hibrid seperti
pembelajaran penguatan.

Dari sudut pandang konseptual, perkuliahan ini mengontraskan gagasan Lamarck tentang karakteristik
yang diperoleh dengan pandangan Darwin-Wallace tentang seleksi. Algoritma genetika
mengikuti prinsip Darwin dengan menerapkan variasi acak dan seleksi untuk mencari
solusi yang baik. Contoh-contoh visual ini mendemonstrasikan penerapan GA yang luas di berbagai bidang hiburan
(AI game dan pembuatan konten prosedural), robotika (perencanaan jalur dan perilaku
adaptif), simulasi (kehidupan buatan dan studi evolusi), optimasi (perencanaan
rute dan alokasi sumber daya), dan penelitian (mempelajari proses evolusioner).
Wawasan kuncinya adalah bahwa GA menyediakan kerangka kerja terpadu untuk menyelesaikan masalah
optimasi kompleks di berbagai domain, menjadikannya alat yang serbaguna dalam
kecerdasan komputasional.