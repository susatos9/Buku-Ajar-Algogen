\chapter{Mutasi dan Pembaruan Generasi}
\label{ch:mutation-update}

Pada bab-bab sebelumnya, kita telah membahas operasi-operasi fundamental Algoritma Genetika (AG) termasuk pengkodean, evaluasi kebugaran, seleksi, dan pindah silang. Bab ini melengkapi pembahasan operator AG dengan mengkaji \textbf{mutasi} dan \textbf{mekanisme pembaruan generasi}~\cite{course_week9, tutorialspoint_mutation}. Operasi-operasi ini sangat penting untuk mempertahankan keragaman genetik dan memastikan kemampuan algoritma untuk mengeksplorasi ruang pencarian secara efektif.

\section{Pengantar Mutasi}
Setelah tahap rekombinasi (pindah silang) diterapkan pada semua pasangan kromosom dalam kumpulan kawin, menghasilkan $N$ kromosom (dengan $N$ adalah ukuran populasi), AG mengeksekusi operator mutasi pada setiap kromosom tersebut. Mutasi adalah operator kritis yang mencegah konvergensi prematur ke optima lokal, mempertahankan keragaman genetik dalam populasi, memperkenalkan materi genetik baru yang mungkin tidak ada dalam populasi awal, dan menyediakan mekanisme untuk meloloskan diri dari optima lokal.

\subsection{Apa itu Mutasi?}

Mutasi adalah proses mengubah nilai satu atau lebih gen dalam genom~\cite{goldberg1989genetic, mitchell1996introduction, back1996evolutionary}. Lebih spesifik, mutasi dapat mengubah alel dari gen pada lokus tertentu ke alel lain, membantu menghindari konvergensi prematur (yaitu mencapai hasil suboptimal yang bukan maksimum global), dan menciptakan keturunan yang belum tentu lebih baik dari induknya.

Secara konkret, mutasi mengubah satu atau lebih nilai gen (alel) pada lokus yang dipilih. Perubahan ini bisa acak atau mengikuti aturan stokastik sederhana; tujuan utamanya adalah mempertahankan variasi dalam populasi sehingga proses pencarian dapat terus mengeksplorasi wilayah-wilayah menjanjikan dan baru dari lanskap kebugaran. Mutasi kadang-kadang menghasilkan keturunan yang lebih rendah, tetapi sering kali merupakan satu-satunya mekanisme yang mampu memperkenalkan blok bangunan baru yang mengarah pada perbaikan di masa depan.

\textbf{Catatan Penting:} Populasi baru yang dihasilkan dari mutasi tidak dijamin lebih baik dari populasi sebelumnya. Namun, mutasi menyediakan mekanisme esensial untuk mempertahankan keragaman dan mengeksplorasi wilayah baru dari ruang pencarian.

\subsection{Mutasi dalam Algoritma Evolusioner vs. Evolusi Biologis}

Dalam evolusi biologis, mutasi biasanya dianggap berbahaya karena organisme kompleks memiliki sistem yang sangat saling bergantung. Namun, dalam Algoritma Evolusioner (AE):

Meskipun mutasi biologis sering merusak pada organisme kompleks, situasinya berbeda dalam Algoritma Evolusioner. Representasi yang digunakan dalam AE biasanya jauh lebih sederhana dan lebih modular daripada genom biologis, sehingga perubahan kecil dan terlokalisasi dapat menghasilkan variasi konstruktif. Sebagai hasilnya, mutasi dalam AE sering dapat menghasilkan keragaman yang bermanfaat: memutasikan subset kecil dari gen dapat menghasilkan keturunan yang lebih baik tanpa mengganggu komponen fungsional lain dari solusi.

\section{Mutasi untuk Representasi Berbeda}

Banyak metode mutasi telah diusulkan dalam literatur~\cite{michalewicz1996genetic, back1996evolutionary, haupt2004practical}. Setiap metode memiliki karakteristik khusus dan mungkin hanya dapat diterapkan pada jenis representasi tertentu. Pemilihan operator mutasi harus kompatibel dengan skema pengkodean kromosom.

\subsection{Mutasi untuk Representasi Biner}
\label{sec:binary-mutation}

Representasi biner menggunakan bentuk mutasi paling sederhana: \textbf{mutasi pembalikan bit}.

\subsubsection{Mutasi Pembalikan Bit}
Dalam mutasi pembalikan bit, setiap bit dalam kromosom memiliki probabilitas $P_m$ (probabilitas mutasi) untuk dibalik: bit dengan nilai $1$ menjadi $0$, dan bit dengan nilai $0$ menjadi $1$.

Dalam pengkodean biner, mutasi paling sederhana adalah pembalikan bit: setiap bit secara independen dibalik dengan probabilitas $P_m$, sehingga 1 menjadi 0 dan sebaliknya. Operator ini minimal dan tidak bias, dan ketika $P_m$ kecil, operator ini memberikan perturbasi langka tetapi bermakna pada string bit yang stabil.

\textbf{Contoh:}
\begin{verbatim}
Induk:     1 0 1 1 0 1 0 0
                 ^     ^
Keturunan: 1 0 0 1 0 0 0 0
\end{verbatim}

Dalam contoh ini, bit pada posisi 3 dan 6 dipilih untuk mutasi dan dibalik.

\textbf{Algoritma:}
\begin{algorithm}[H]
\caption{Mutasi Pembalikan Bit}
\begin{algorithmic}
\FOR{setiap gen $g_i$ dalam kromosom}
    \STATE $r \gets$ bilangan acak dalam $[0,1]$
    \IF{$r < P_m$}
        \STATE Balik $g_i$: jika $g_i = 1$ maka $g_i \gets 0$, sebaliknya $g_i \gets 1$
    \ENDIF
\ENDFOR
\end{algorithmic}
\end{algorithm}

\subsection{Mutasi untuk Representasi Integer}
\label{sec:integer-mutation}

Representasi integer memerlukan strategi mutasi yang berbeda. Pendekatan umum meliputi pembalikan nilai integer, pemilihan nilai acak, dan mutasi creep.

\subsubsection{Pembalikan Nilai Integer}

Menggunakan operasi matematika ($+$, $-$, $\times$, $\div$) untuk mengubah nilai gen yang dipilih.

\textbf{Contoh:}
\begin{verbatim}
Induk:     8  3  7  5  2  1  9  4  6
                 ^        ^
Keturunan: 8  3  2  5  2  8  9  4  6
\end{verbatim}

Nilai pada posisi 3 dan 6 diubah menggunakan operasi matematika.

\subsubsection{Pemilihan Nilai Acak}

Gen yang dipilih diganti dengan nilai yang dipilih secara acak dari rentang yang valid.

\textbf{Contoh:}
Jika rentang yang valid adalah $[1, 9]$:
\begin{verbatim}
Induk:     8  3  7  5  2  1  9  4  6
                    ^
Keturunan: 8  3  7  9  2  1  9  4  6
\end{verbatim}

\subsubsection{Mutasi Creep}

Menambahkan atau mengurangkan nilai integer acak kecil (biasanya $\pm 1$ atau $\pm 2$) pada gen yang dipilih.

\textbf{Contoh:}
\begin{verbatim}
Induk:     8  3  7  5  2  1  9  4  6
              ^           ^
Keturunan: 8  4  7  5  2  2  9  4  6
\end{verbatim}

Metode ini membuat perubahan kecil dan bertahap serta sangat berguna untuk penyetelan halus solusi.

\subsection{Mutasi untuk Representasi Bernilai Riil}
\label{sec:real-mutation}

Representasi bernilai riil memiliki karakteristik berbeda dari representasi biner dan integer. Nilai gen dalam representasi riil bersifat kontinu, sedangkan representasi biner dan integer bersifat diskrit. Oleh karena itu, representasi riil memerlukan operator mutasi khusus.

\subsubsection{Mutasi Seragam}
Dalam mutasi seragam, gen yang dipilih diganti dengan nilai yang diambil dari distribusi acak seragam dalam rentang valid $[a, b]$:

\begin{equation}
x_i' = a + \text{rand}(0,1) \times (b - a)
\end{equation}

dengan:
\begin{itemize}
    \item $x_i'$ adalah nilai gen baru
    \item $a$ dan $b$ adalah batas bawah dan atas
    \item $\text{rand}(0,1)$ menghasilkan bilangan acak dalam $[0, 1]$
\end{itemize}

\subsubsection{Mutasi Tidak Seragam dengan Distribusi Tetap}

Mutasi ini mirip dengan metode creep untuk representasi integer tetapi menggunakan penambahan bernilai riil. Nilai yang bermutasi dihitung sebagai:

\begin{equation}
x_i' = x_i + \mathcal{N}(0, \sigma^2)
\end{equation}

dengan:
\begin{itemize}
    \item $x_i$ adalah nilai gen asli
    \item $\mathcal{N}(0, \sigma^2)$ adalah nilai acak dari distribusi normal (Gaussian) dengan mean 0 dan variansi $\sigma^2$
    \item $\sigma$ mengontrol ukuran langkah mutasi
\end{itemize}

\textbf{Contoh:}
\begin{verbatim}
Induk:     2.45  7.89  3.12  9.01  5.67
                       ^
Keturunan: 2.45  7.89  3.45  9.01  5.67
\end{verbatim}

\subsection{Mutasi untuk Representasi Permutasi}
\label{sec:permutation-mutation}

Mutasi pada representasi permutasi harus memastikan bahwa kromosom yang dihasilkan tetap valid (semua elemen muncul tepat sekali). Metode khusus telah dikembangkan untuk menjaga validitas sambil memperkenalkan variasi.

\subsubsection{Mutasi Tukar}

Dua posisi gen dipilih secara acak, dan nilainya ditukar.

\textbf{Contoh:}
\begin{verbatim}
Induk:     3  1  5  2  7  6  8  4  9
              ^           ^
Keturunan: 3  1  8  2  7  6  5  4  9
\end{verbatim}

Posisi 3 dan 7 dipilih, sehingga nilai 5 dan 8 ditukar.

\textbf{Algoritma:}
\begin{algorithm}[H]
\caption{Mutasi Tukar}
\begin{algorithmic}
\STATE $i \gets$ posisi acak dalam kromosom
\STATE $j \gets$ posisi acak dalam kromosom (berbeda dari $i$)
\STATE Tukar nilai pada posisi $i$ dan $j$
\end{algorithmic}
\end{algorithm}

\subsubsection{Mutasi Sisip}

Gen pada satu posisi dihapus dan disisipkan pada posisi lain, menggeser gen-gen di antara keduanya.

\textbf{Contoh:}
\begin{verbatim}
Induk:     3  1  5  2  7  6  8  4  9
              ^           ^
Keturunan: 3  1  5  2  7  8  6  4  9
\end{verbatim}

Gen pada posisi 7 (nilai 8) dihapus dan disisipkan setelah posisi 2 (nilai 5).

\subsubsection{Mutasi Acak}

Segmen kromosom dipilih, dan gen-gen dalam segmen tersebut diacak secara acak.

\textbf{Contoh:}
\begin{verbatim}
Induk:     3  1  5  2  7  6  8  4  9
              \_______/
Keturunan: 3  1  2  6  5  7  8  4  9
\end{verbatim}

Segmen $\{5, 2, 7, 6\}$ dipilih dan diacak secara acak menjadi $\{2, 6, 5, 7\}$.

\subsubsection{Mutasi Inversi}

Segmen kromosom dipilih, dan urutan gen dalam segmen tersebut dibalik.

\textbf{Contoh:}
\begin{verbatim}
Induk:     3  1  5  2  7  6  8  4  9
              \_______/
Keturunan: 3  1  6  7  2  5  8  4  9
\end{verbatim}

Segmen $\{5, 2, 7, 6\}$ dibalik menjadi $\{6, 7, 2, 5\}$.

\section{Mekanisme Pembaruan Generasi}

Setelah operasi seleksi, pindah silang, dan mutasi diterapkan pada populasi, mekanisme pembaruan generasi menentukan individu mana yang bertahan ke generasi berikutnya. Proses ini juga disebut \textbf{seleksi penyintas} atau \textbf{strategi penggantian}.

\subsection{Model Asli Holland (Penggantian Generasional)}
\label{sec:holland-update}
Dalam AG asli Holland~\cite{holland1975adaptation, goldberg1989genetic}, semua keturunan menggantikan seluruh populasi induk. Induk dianggap "mati" dan dihapus, sehingga populasi baru sepenuhnya terdiri dari keturunan dan generasi bersifat berbeda dan tidak tumpang tindih.

Dalam model penggantian generasional asli Holland, populasi keturunan sepenuhnya menggantikan induk, menghasilkan generasi yang berbeda dan tidak tumpang tindih. Model ini sederhana dan mudah diimplementasikan serta memberikan pemisahan yang jelas antar generasi. Kelemahan praktis adalah potensi kehilangan induk berkualitas tinggi kecuali mekanisme seperti elitisme digunakan untuk melestarikannya.

\subsection{Model Generasional dengan Elitisme}
\label{sec:elitism}

Dalam model generasional dengan elitisme, populasi berukuran $N$ kromosom dalam satu generasi digantikan oleh $N$ individu baru di generasi berikutnya~\cite{de1975analysis, whitley1994genetic}. Namun, untuk melestarikan solusi terbaik, $k$ kromosom terbaik (elit) dari generasi induk disalin langsung ke generasi berikutnya sementara $N-k$ posisi yang tersisa diisi dengan keturunan; ini memastikan bahwa solusi terbaik tidak pernah menjadi lebih buruk lintas generasi.

Dalam model generasional dengan elitisme, $k$ individu teratas dari generasi induk dibawa maju tanpa perubahan dan $N-k$ posisi yang tersisa diisi oleh keturunan yang baru dihasilkan. Modifikasi sederhana ini menjamin bahwa solusi terbaik-sejauh-ini tidak hilang, yang menstabilkan pencarian dan sering mempercepat konvergensi. Pilihan $k$ yang khas adalah kecil (misalnya 1 atau 2), menyeimbangkan pelestarian dan eksplorasi.

\textbf{Algoritma:}
\begin{algorithm}[H]
\caption{Model Generasional dengan Elitisme}
\begin{algorithmic}
\STATE Urutkan populasi induk berdasarkan kebugaran
\STATE Salin $k$ individu teratas ke generasi berikutnya (elit)
\STATE Hasilkan $N-k$ keturunan melalui seleksi, pindah silang, dan mutasi
\STATE Tambahkan keturunan ke generasi berikutnya
\STATE Generasi berikutnya menjadi generasi saat ini
\end{algorithmic}
\end{algorithm}

\textbf{Nilai khas:} $k = 1$ atau $k = 2$ (melestarikan 1-2 individu terbaik)

\subsection{Pembaruan Steady-State}
\label{sec:steady-state}

Dalam model steady-state~\cite{whitley1994genetic, smith1998replacement}, tidak semua kromosom digantikan dalam setiap generasi; hanya $M$ kromosom yang digantikan dengan $M < N$ (sering $M = 2$, ketika satu perkawinan menghasilkan dua keturunan yang menggantikan dua individu). Strategi penggantian meliputi: \textbf{ganti induk} (dua keturunan menggantikan dua induknya), \textbf{ganti terburuk} (dua keturunan menggantikan dua individu terburuk), dan \textbf{ganti tertua} (dua keturunan menggantikan dua individu tertua). Model ini memungkinkan individu baik untuk berpartisipasi dalam beberapa perkawinan, menghasilkan evolusi yang lebih bertahap, memungkinkan induk dan keturunan hidup berdampingan dalam populasi yang sama, dan dapat lebih efisien secara komputasional.

Dalam skema pembaruan steady-state, hanya sejumlah kecil $M$ individu (dengan $M<N$) yang digantikan pada setiap langkah, yang memungkinkan induk dan keturunan hidup berdampingan dan memungkinkan individu berkualitas tinggi digunakan kembali dalam beberapa perkawinan. Strategi penggantian umum adalah menggantikan induk dari keturunan, menggantikan individu terburuk yang ditemukan dalam populasi, atau menggantikan individu tertua; setiap strategi menekankan trade-off berbeda antara melestarikan keragaman dan mengintensifkan seleksi. Pendekatan steady-state biasanya menghasilkan evolusi yang lebih bertahap dan dapat efisien secara komputasional ketika $M$ kecil.

\subsection{Pembaruan Kontinu}
\label{sec:continuous-update}

Dalam pembaruan kontinu, keturunan dan induk dapat hidup berdampingan dalam generasi yang sama; individu dipilih secara acak dari kedua kelompok untuk generasi berikutnya, memberikan tumpang tindih maksimum antar generasi. Metode ini kurang umum digunakan dibandingkan metode pembaruan lainnya.

Skema pembaruan kontinu memungkinkan koeksistensi penuh antara induk dan keturunan dan biasanya memilih individu untuk bertahan dari set gabungan. Ini menghasilkan tumpang tindih generasional maksimum dan populasi yang sangat bercampur, meskipun dalam praktiknya skema seperti itu kurang umum digunakan dibandingkan dengan penggantian generasional atau steady-state.

\section{Parameter AG}

Kinerja Algoritma Genetika sangat bergantung pada pengaturan parameter yang tepat~\cite{grefenstette1986optimization, schaffer1989study, de1975analysis}. Parameter utama yang perlu dikonfigurasi adalah:

\subsection{Probabilitas Pindah Silang ($P_c$)}
\label{sec:crossover-probability}

$P_c$ adalah probabilitas bahwa dua induk akan mengalami pindah silang. Jika $P_c=100\%$, semua keturunan dihasilkan melalui pindah silang; jika $P_c=0\%$, tidak ada pindah silang yang terjadi dan keturunan adalah salinan persis dari induk. Nilai khas berada dalam rentang $P_c\in[0.65,0.90]$. Nilai yang lebih tinggi (0.8–0.9) mendorong eksplorasi, sedangkan nilai yang lebih rendah melestarikan solusi baik tetapi mengurangi keragaman; pengaturan awal standar adalah $P_c=0.8$.

Probabilitas pindah silang $P_c$ mengontrol seberapa sering rekombinasi terjadi. Nilai mendekati 1 (misalnya 0.8–0.9) mendorong pencampuran agresif materi parental dan oleh karena itu eksplorasi ruang pencarian, sedangkan nilai yang lebih rendah mengkonservasi struktur parental dan memperlambat penciptaan kombinasi baru. Default yang umum digunakan adalah $P_c\approx0.8$, tetapi pilihan akhir tergantung pada karakteristik masalah dan penyetelan empiris.

\subsection{Probabilitas Mutasi ($P_m$)}
\label{sec:mutation-probability}

$P_m$ adalah probabilitas bahwa gen dalam kromosom keturunan akan mengalami mutasi. Ketika $P_m=100\%$, semua gen bermutasi (menyebabkan kekacauan), dan ketika $P_m=0\%$, tidak ada mutasi yang terjadi dan tidak ada materi genetik baru yang diperkenalkan. Nilai khas adalah kecil, misalnya $P_m\in[0.005,0.01]$ (0.5%–1%). Formula umum adalah

Probabilitas mutasi khas sangat kecil sehingga mutasi terjadi jarang; nilai dalam rentang $0.5\%$ hingga $1\%$ per gen adalah titik awal yang umum. Dua heuristik yang umum digunakan adalah $P_m=1/L$ (satu mutasi per kromosom rata-rata) atau $P_m=1/(N\times L)$ ketika menskalakan mutasi relatif terhadap total evaluasi. Mengatur $P_m$ terlalu tinggi merusak struktur yang berguna, sedangkan mengaturnya terlalu rendah dapat memungkinkan konvergensi prematur melalui hilangnya keragaman.

\begin{equation}
P_m = \frac{1}{L}
\end{equation}
atau
\begin{equation}
P_m = \frac{1}{N \times L}
\end{equation}

dengan:
\begin{itemize}
    \item $L$ adalah panjang kromosom (jumlah gen)
    \item $N$ adalah ukuran populasi
\end{itemize}

\textbf{Alasan:} Probabilitas mutasi sering diatur sehingga, rata-rata, satu mutasi terjadi per kromosom.

\subsection{Ukuran Populasi ($N$)}
\label{sec:population-size}

Ukuran populasi harus proporsional dengan volume ruang pencarian. Jika populasi terlalu kecil, mungkin sulit mencapai optimum global dan pencarian dapat konvergen ke optima lokal; jika populasi terlalu besar, ini memberlakukan biaya komputasi yang berat dan dapat tidak perlu. Rentang khas adalah $N\in[50,100]$, tetapi nilai yang tepat harus ditentukan melalui eksperimen dan dipilih sesuai dengan kompleksitas masalah dan sumber daya komputasi yang tersedia.

Ukuran populasi $N$ memediasi trade-off antara cakupan eksplorasi ruang pencarian dan biaya komputasi. Populasi kecil dapat gagal mewakili keragaman yang cukup dan dapat konvergen ke optima lokal, sedangkan populasi yang terlalu besar meningkatkan waktu eksekusi tanpa keuntungan proporsional. Sebagai panduan praktis, banyak masalah dimulai dengan $N$ antara 50 dan 100 dan kemudian menyesuaikan berdasarkan kinerja empiris dan sumber daya komputasi yang tersedia.

\subsection{Jumlah Generasi ($G$)}
\label{sec:num-generations}

Jumlah generasi harus proporsional dengan ukuran populasi dan ukuran ruang pencarian.

Jumlah generasi $G$ harus dipilih dalam kaitannya dengan $N$ dan kompleksitas ruang pencarian: masalah yang lebih besar atau lebih kompleks biasanya memerlukan lebih banyak generasi untuk konvergen. Kriteria penghentian umum meliputi jumlah generasi tetap, jumlah evaluasi kebugaran maksimum, tidak ada perbaikan selama $k$ generasi berturut-turut, mencapai kebugaran target, atau kombinasi yang sesuai dari kondisi-kondisi ini.

\subsection{Pedoman Umum Pengaturan Parameter}
\label{sec:parameter-guidelines}

\textbf{Catatan Penting:} Tidak ada aturan universal untuk memilih parameter AG~\cite{wolpert1997no, grefenstette1986optimization}. Pengaturan yang baik biasanya ditemukan melalui kombinasi heuristik teoritis, pengalaman sebelumnya, dan eksperimen sistematis. Konfigurasi awal yang masuk akal adalah memilih representasi yang sesuai dengan masalah (biner, integer, riil atau permutasi), mengatur ukuran populasi $N$ dalam puluhan hingga ratusan rendah (misalnya 50–100), menggunakan $P_c\approx0.8$, dan mengatur heuristik mutasi seperti $P_m\approx1/L$ (atau varian berskala seperti $1/(N\times L)$) dengan penyetelan selanjutnya berdasarkan hasil.

\section{Studi Observasi Parameter}

Untuk memahami efek dari parameter yang berbeda, kami menyajikan studi observasi sistematis.

\subsection{Masalah Uji}

\textbf{Tujuan:} Meminimalkan fungsi:
\begin{equation}
h(x_1, x_2) = x_1^2 + x_2^2
\end{equation}

dengan $x_1, x_2 \in [-10, 10]$

\textbf{Fungsi kebugaran:}
\begin{equation}
\text{Kebugaran} = \frac{1}{x_1^2 + x_2^2 + 0.001}
\end{equation}

Konstanta 0.001 ditambahkan untuk menghindari pembagian dengan nol pada titik optimal $(0, 0)$.

\subsection{Pengaturan Eksperimental}
\textbf{Pengaturan eksperimental:} Studi memvariasikan ukuran populasi (50, 100, 200), presisi bit per variabel (10, 50, 90), probabilitas pindah silang ($P_c\in\{0.5,0.7,0.9\}$), dan probabilitas mutasi relatif terhadap panjang kromosom (misalnya $0.5/L,\;1/L,\;2/L$). Untuk memastikan perbandingan yang adil, setiap konfigurasi dibatasi oleh maksimum 20.000 individu yang dievaluasi dan diulang 30 kali untuk mendapatkan statistik yang andal.

\subsection{Hasil Sampel}

Tabel~\ref{tab:ga-parameters} menunjukkan hasil yang dipilih dari studi parameter:

\begin{table}[H]
\centering
\caption{Hasil Observasi Parameter AG}
\label{tab:ga-parameters}
\begin{tabular}{cccccc}
\toprule
\textbf{Ukuran Pop} & \textbf{Bit} & \textbf{$P_c$} & \textbf{$P_m$} & \textbf{Rata-rata Kebugaran Terbaik} & \textbf{Rata-rata Evaluasi} \\
\midrule
50  & 10 & 0.5 & 0.0250 & 839.55 & 20000 \\
50  & 50 & 0.5 & 0.0050 & 1000.00 & 8301.67 \\
50  & 50 & 0.7 & 0.0100 & 1000.00 & 20000 \\
50  & 90 & 0.7 & 0.0056 & 1000.00 & 8780.00 \\
100 & 50 & 0.7 & 0.0050 & 1000.00 & 14416.67 \\
100 & 90 & 0.5 & 0.0111 & 1000.00 & 20000 \\
200 & 50 & 0.5 & 0.0050 & 1000.00 & 20000 \\
200 & 90 & 0.7 & 0.0056 & 1000.00 & 20000 \\
200 & 90 & 0.9 & 0.0028 & 1000.00 & 19866.67 \\
\bottomrule
\end{tabular}
\end{table}

\textbf{Pengamatan kunci:} Konfigurasi paling efisien dalam eksperimen ini adalah ukuran populasi 50 dengan 90 bit per variabel, $P_c = 0.7$ dan $P_m \approx 0.0056$, yang secara konsisten mencapai optimum (fitness 1000.00) sambil hanya memerlukan sekitar 8780 evaluasi rata-rata. Mengenai presisi, 10 bit sering kali tidak cukup untuk mencapai optimum, sedangkan 50–90 bit memberikan granularitas yang diperlukan untuk konvergensi yang andal. Populasi yang lebih kecil (misalnya 50) terbukti efisien dalam masalah uji ini, sedangkan populasi yang lebih besar (misalnya 200) menawarkan lebih banyak ketahanan dengan biaya komputasi yang lebih besar — sebuah trade-off klasik antara kecepatan dan keandalan. Probabilitas pindah silang sekitar $0.7$ cenderung menyeimbangkan eksplorasi dan eksploitasi secara efektif. Akhirnya, tingkat mutasi rendah pada orde $1/L$ bekerja paling baik: tingkat yang terlalu tinggi memperkenalkan keacakan yang mengganggu, sementara tingkat yang terlalu rendah mengurangi keragaman dan meningkatkan risiko konvergensi prematur.

\section{Latihan}
\begin{enumerate}
\item Diberikan dua kromosom induk untuk masalah permutasi:
\begin{itemize}
\item Induk 1: [1, 2, 7, 3, 4, 9, 8, 6, 5]
\item Induk 2: [5, 4, 3, 9, 1, 2, 6, 8, 7]
\end{itemize}
\begin{enumerate}
\item Lakukan Partial-Mapped Crossover (PMX) dengan titik potong pada posisi 2 dan 5
\item Terapkan mutasi inversi pada keturunan dengan segmen mutasi dari lokus 2 hingga 5
\end{enumerate}
\item Untuk GA dengan pengkodean biner dengan panjang kromosom $L = 50$ dan ukuran populasi $N = 100$:
\begin{enumerate}
    \item Hitung probabilitas mutasi yang sesuai menggunakan $P_m = 1/L$
    \item Hitung probabilitas mutasi alternatif menggunakan $P_m = 1/(N \times L)$
    \item Diskusikan mana yang mungkin lebih sesuai dan mengapa
\end{enumerate}

\item Rancang operator mutasi untuk kromosom bernilai riil yang mewakili koordinat $(x, y)$ di mana $x, y \in [-100, 100]$:
\begin{enumerate}
    \item Implementasikan mutasi seragam
    \item Implementasikan mutasi Gaussian dengan $\sigma = 5$
    \item Bandingkan perilaku yang diharapkan dari kedua operator
\end{enumerate}

\item Implementasikan dan bandingkan tiga strategi pembaruan generasi:
\begin{enumerate}
    \item Penggantian generasional dengan elitisme ($k=2$)
    \item Steady-state dengan penggantian individu terburuk
    \item Steady-state dengan penggantian individu tertua
\end{enumerate}
Diskusikan skenario di mana masing-masing mungkin lebih disukai.

\item Untuk fungsi uji $f(x_1, x_2) = x_1^2 + x_2^2$ dengan $x_1, x_2 \in [-10, 10]$:
\begin{enumerate}
    \item Rancang GA lengkap termasuk semua parameter
    \item Jalankan eksperimen dengan kombinasi parameter yang berbeda
    \item Analisis parameter mana yang memiliki dampak paling signifikan
    \item Usulkan konfigurasi parameter optimal berdasarkan hasil Anda
\end{enumerate}
\end{enumerate}